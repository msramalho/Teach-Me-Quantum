\documentclass[aspectratio=43]{beamer}
% \documentclass[handout]{beamer}
\usepackage[utf8]{inputenc}
\usepackage{multicol}
\usepackage{tikz} % drawings
\usepackage{animate} % animations
\usepackage{hyperref}
\usepackage{macros}

\usetikzlibrary{positioning, arrows}

\graphicspath{{img/}} % add the img folder to graphics path

\title{Hello \qw}
\date{November, 2018}
\author[Ramalho]{Miguel Sozinho Ramalho}
% \institution[FEUP]{Faculty of Engineering of the University of Porto}


%%%%%%%%%%%%%%%%%%%%%%%% THEME
\usetheme{material}
\useLightTheme
\usePrimaryTeal
\useAccentGreen

\usepackage{pgfpages}
% \setbeameroption{show notes}
% \setbeameroption{show notes on second screen=right}

\begin{document}

\begin{frame}
	\titlepage
\end{frame}


\begin{frame}{Table of contents}
	\begin{card}
		\tableofcontents
	\end{card}
\end{frame}

% Topics
% Theory
% Exercises explanation
% Conclusion
% References and additional sources

\section{Motivation}
\begin{frame}{Motivation}
\begin{card}[Why]
    \begin{chapquote}[2pt]{\href{https://en.wikipedia.org/wiki/Seth_Lloyd}{Seth Lloyd}}
        ``Classical computation is like a solo voice - one line of pure tones succeeding each other. Quantum computation is like a symphony - many lines of tones interfering with one another.''
    \end{chapquote}
\end{card}

\pagenumber
\end{frame}

\begin{frame}{Motivation}
\begin{card}[How]
    \qc can be seen as leveraging the phenomena that happen at the atomic and subatomic levels - in the \qw\xspace- to produce computations that, ultimately, surpass \cc.
\end{card}
\pagenumber
\end{frame}

\begin{frame}{About this course}
\begin{card}
    This course is suited for beginners in \qm and \qc. If you are already familiar with the concepts of a given week, you are encouraged to move forward on the course. 
    
    This course brings novelty in that it focuses on \textbf{learning by doing}, and that is why you will also learn about \qk and \ibmqe.
\end{card}
\pagenumber
\end{frame}

\begin{frame}{About this course - Study plan}
\begin{card}
    \begin{itemize}
        \item \qm 101
        \item \qk and \ibmq
        \item \qi
        \item Designing \qcts
        \item \qa
        \item \qc applications
        \item \q Computers - state of the art
        \item History, implications and ethics
    \end{itemize}
\end{card}
\pagenumber
\end{frame}

\section{\qp vs \cp}

\begin{frame}{\cp}
\begin{card}
    \cp (also \cm) describes the world \textit{as we see it}, in its macro level. Some of its properties are:
    \begin{description}
        \item[size] objects with $size \gtrsim 1nm\ (10^{-9}m)$
        \item[speed] objects of $speed \lesssim \speedoflight$
        \item[causality] knowledge of the past allows calculation of the future (and \textit{vice versa})
    \end{description}
\end{card}
\pagenumber
\end{frame}


\begin{frame}{\qp}
\begin{card}
    \qp (also \qm) describes the world \textit{as we see it}, in its macro level. Some of its properties are:
    \begin{description}
        \item[size] objects with $size \lesssim 1nm$
        \item[speed] objects of $speed \lesssim \speedoflight$
        \item[causality] knowledge of the past allows calculation of the future (and vice-versa)
    \end{description}
\end{card}
\pagenumber
\end{frame}

\begin{frame}{\qp}
\begin{card}[The kingdoms of \cl and \qm]
    \centering\cardImg{classical_vs_quantum_dimensions.png}{\textwidth}
\end{card}
\pagenumber
\end{frame}



\section{References}
\begin{frame}{References}
\begin{card}
    \begin{itemize}
        \item \href{https://www.goodreads.com/book/show/331680.Programming_the_Universe}{Programming the Universe: A Quantum Computer Scientist Takes on the Cosmos}
    \end{itemize}
\end{card}
\end{frame}



\begin{frame}
	\begin{equation}
	    \superpos[\delta]
	\end{equation}
	\begin{equation}
		\superpos[\phi][-][+]
	\end{equation}
	\begin{equation}
		\plusminus
	\end{equation}
	\par
\end{frame}

\begin{frame}
	\begin{card}
		Material Design theme is a theme for Beamer inspired by Google's Material Design. \\[5mm]
		This manual only covers the theme itself, for more information on Material Design go to:
		\\\url{https://material.io}
	\end{card}
	\note[item]{Thank the audience for being awake.}
\end{frame}


\end{document}
